% kickoff.tex — UNISTRA NLP 2026 Kickoff Presentation
% Compile with: xelatex kickoff.tex (requires fontspec)
% Or: pdflatex kickoff.tex (will use default fonts)

\documentclass[aspectratio=169,12pt]{beamer}

% Try to use the custom theme; fallback gracefully
\IfFileExists{beamerthemeUNISTRA.sty}{%
  \usetheme{UNISTRA}%
}{%
  \usetheme{default}%
  \usepackage{xcolor}%
  \definecolor{bgdark}{HTML}{2C2421}%
  \definecolor{cream}{HTML}{FAF7F2}%
  \definecolor{coral}{HTML}{E07850}%
  \definecolor{gold}{HTML}{D4A855}%
  \definecolor{textmuted}{HTML}{A89B91}%
  \setbeamercolor{background canvas}{bg=bgdark}%
  \setbeamercolor{normal text}{fg=cream}%
  \setbeamercolor{structure}{fg=coral}%
  \setbeamertemplate{navigation symbols}{}%
  \newcommand{\highlight}[1]{\textcolor{coral}{#1}}%
  \newcommand{\muted}[1]{\textcolor{textmuted}{#1}}%
  \newcommand{\gold}[1]{\textcolor{gold}{#1}}%
}

\usepackage{graphicx}
\usepackage{hyperref}
\usepackage{booktabs}
\usepackage{tikz}
\usetikzlibrary{shapes,arrows.meta,positioning,calc}

% ── Metadata ──
\title{Applied NLP}
\subtitle{From Text to Intelligence}
\author{Roman Jurowetzki}
\institute{Aalborg University / University of Strasbourg}
\date{February 10--12, 2026}

\begin{document}

% ═══════════════════════════════════════
% SLIDE 1: Title
% ═══════════════════════════════════════
\begin{frame}[plain]
  \titlepage
\end{frame}

% ═══════════════════════════════════════
% SLIDE 2: About Me
% ═══════════════════════════════════════
\begin{frame}{About Me}
  \begin{columns}[T]
    \begin{column}{0.6\textwidth}
      \textbf{Roman Jurowetzki}\\[0.3cm]
      \muted{Associate Professor}\\
      Aalborg University Business School\\[0.3cm]
      \muted{Research:}\\
      NLP for innovation studies, science policy,\\
      LLMs for social science research\\[0.3cm]
      \muted{Teaching:}\\
      Business Data Science, Applied NLP,\\
      AI for Social Scientists
    \end{column}
    \begin{column}{0.35\textwidth}
      \begin{center}
        \muted{Find me:}\\[0.2cm]
        \small
        \highlight{@rjuro} on GitHub\\
        \highlight{@rjuro} on Twitter/X\\
        rjuro@business.aau.dk
      \end{center}
    \end{column}
  \end{columns}
\end{frame}

% ═══════════════════════════════════════
% SLIDE 3: The NLP Task Landscape
% ═══════════════════════════════════════
\begin{frame}{The NLP Task Landscape}
  \framesubtitle{What can we do with text?}

  \begin{center}
  \begin{tikzpicture}[
    node distance=1.2cm,
    task/.style={rectangle, draw=coral, fill=bgdark, text=cream, minimum width=3cm, minimum height=0.8cm, font=\small, line width=0.5pt},
    label/.style={font=\footnotesize, text=textmuted}
  ]
    \node[task] (class) {Classification};
    \node[task, right=1.5cm of class] (extract) {Extraction};
    \node[task, right=1.5cm of extract] (sim) {Similarity};
    \node[task, below=0.8cm of class] (topic) {Topic Modeling};
    \node[task, below=0.8cm of extract] (gen) {Generation};
    \node[task, below=0.8cm of sim] (qa) {Question Answering};

    \node[label, below=0.15cm of class] {sentiment, stance, spam};
    \node[label, below=0.15cm of extract] {NER, relations, JSON};
    \node[label, below=0.15cm of sim] {search, matching};
    \node[label, below=0.15cm of topic] {BERTopic, LDA};
    \node[label, below=0.15cm of gen] {summaries, translation};
    \node[label, below=0.15cm of qa] {RAG, chatbots};
  \end{tikzpicture}
  \end{center}

  \vspace{0.3cm}
  \muted{This workshop covers all of these --- from simple baselines to LLM-powered pipelines.}
\end{frame}

% ═══════════════════════════════════════
% SLIDE 4: The 5 Eras of Text Representation
% ═══════════════════════════════════════
\begin{frame}{The 5 Eras of Text Representation}
  \framesubtitle{A brief history of how computers read text}

  \begin{center}
  \begin{tikzpicture}[
    era/.style={rectangle, draw=coral, fill=bgdark, text=cream, minimum width=2.2cm, minimum height=0.7cm, font=\footnotesize\bfseries, line width=0.5pt},
    year/.style={font=\tiny, text=gold},
    desc/.style={font=\tiny, text=textmuted, text width=2cm, align=center}
  ]
    % Timeline line
    \draw[coral, line width=1pt] (0,0) -- (12,0);

    % Era markers
    \foreach \x/\yr/\name/\detail in {
      0.5/1950s/Regex/Rules \& patterns,
      3/1990s/TF-IDF/Sparse vectors,
      5.5/2013/Word2Vec/Dense embeddings,
      8/2018/BERT/Contextual,
      10.5/2020+/LLMs/Foundation models
    } {
      \fill[coral] (\x, 0) circle (3pt);
      \node[era, above=0.4cm] at (\x, 0) {\name};
      \node[year, below=0.15cm] at (\x, 0) {\yr};
      \node[desc, below=0.55cm] at (\x, 0) {\detail};
    }
  \end{tikzpicture}
  \end{center}

  \vspace{0.5cm}
  Each era \highlight{didn't replace} the previous one --- it \highlight{added a new tool} to the toolkit.\\[0.2cm]
  \muted{TF-IDF still wins sometimes. The right tool depends on your task.}
\end{frame}

% ═══════════════════════════════════════
% SLIDE 5: Where We Are Now
% ═══════════════════════════════════════
\begin{frame}{Where We Are Now}
  \framesubtitle{The foundation model era}

  \begin{itemize}
    \item \highlight{Embeddings are everywhere} --- sentence, document, image, code
    \item \highlight{Few-shot learning} --- classify with 8 examples, not 8,000
    \item \highlight{Structured output} --- LLMs return JSON, not just text
    \item \highlight{Knowledge distillation} --- big model teaches small model
    \item \highlight{Open models} --- Llama, Qwen, Mistral --- run locally
  \end{itemize}

  \vspace{0.5cm}
  \begin{alertblock}{Key Insight}
    The cost of NLP went from ``PhD + 6 months'' to ``API call + 5 minutes.''\\
    But \highlight{evaluation} and \highlight{judgment} matter more than ever.
  \end{alertblock}
\end{frame}

% ═══════════════════════════════════════
% SLIDE 6: Course Roadmap
% ═══════════════════════════════════════
\begin{frame}{Course Roadmap}
  \framesubtitle{20 hours across 3 days}

  \begin{center}
  \begin{tikzpicture}[
    block/.style={rectangle, draw=coral, fill=bgdark, text=cream, minimum width=2.8cm, minimum height=1.8cm, font=\footnotesize, align=center, line width=0.5pt},
    day/.style={font=\small\bfseries, text=gold}
  ]
    % Day labels
    \node[day] at (1.5, 2.2) {Tuesday};
    \node[day] at (5.5, 2.2) {Wednesday};
    \node[day] at (9.5, 2.2) {Thursday};

    % Blocks
    \node[block] at (0, 0.5) {Block 1\\[0.1cm]\tiny TF-IDF +\\Embeddings};
    \node[block] at (3, 0.5) {Block 2\\[0.1cm]\tiny LLMs +\\BERTopic};
    \node[block] at (5.5, 0.5) {Block 3\\[0.1cm]\tiny SetFit +\\FAISS};
    \node[block] at (8, 0.5) {Block 4\\[0.1cm]\tiny Reranking +\\Fine-tuning};
    \node[block] at (11, 0.5) {Block 5\\[0.1cm]\tiny Eval +\\Demos};

    % Sprint markers
    \node[font=\tiny, text=gold] at (3, -0.8) {Sprint 1};
    \node[font=\tiny, text=gold] at (11, -0.8) {Sprint 2};

    % Arrows
    \draw[-{Stealth}, coral, thick] (1.5, 0.5) -- (1.8, 0.5);
    \draw[-{Stealth}, coral, thick] (4.4, 0.5) -- (4.6, 0.5);
    \draw[-{Stealth}, coral, thick] (6.6, 0.5) -- (7, 0.5);
    \draw[-{Stealth}, coral, thick] (9.5, 0.5) -- (9.8, 0.5);
  \end{tikzpicture}
  \end{center}

  \vspace{0.3cm}
  \muted{Each block: guided notebooks + hands-on exercises + comparison to prior approaches.}
\end{frame}

% ═══════════════════════════════════════
% SLIDE 7: The 11 Notebooks
% ═══════════════════════════════════════
\begin{frame}{The 11 Notebooks}
  \footnotesize
  \begin{columns}[T]
    \begin{column}{0.48\textwidth}
      \highlight{Tuesday}\\[0.2cm]
      \begin{tabular}{@{}ll@{}}
        NB01 & TF-IDF + Linear Models \\
        NB02 & Sentence Embeddings \\
        NB03 & LLM Zero-shot \\
        NB04 & BERTopic \\
      \end{tabular}

      \vspace{0.5cm}
      \highlight{Wednesday}\\[0.2cm]
      \begin{tabular}{@{}ll@{}}
        NB05 & SetFit Few-shot \\
        NB06 & FAISS Retrieval \\
      \end{tabular}
    \end{column}

    \begin{column}{0.48\textwidth}
      \highlight{Thursday AM}\\[0.2cm]
      \begin{tabular}{@{}ll@{}}
        NB07 & Cross-encoder Reranking \\
        NB08 & LLM Distillation \\
        NB09 & Fine-tuning (Qwen3-4B) \\
      \end{tabular}

      \vspace{0.5cm}
      \highlight{Thursday PM}\\[0.2cm]
      \begin{tabular}{@{}ll@{}}
        NB10 & LLM Evaluation \\
        NB11 & Annotation \& IRR \\
      \end{tabular}
    \end{column}
  \end{columns}

  \vspace{0.4cm}
  \muted{All notebooks run in Google Colab. No local setup required.}
\end{frame}

% ═══════════════════════════════════════
% SLIDE 8: Tools & Setup
% ═══════════════════════════════════════
\begin{frame}{Tools \& Setup}
  \framesubtitle{Everything is free}

  \begin{columns}[T]
    \begin{column}{0.45\textwidth}
      \highlight{LLM Providers}\\[0.3cm]
      \begin{itemize}
        \item \textbf{Groq} \muted{(primary)}\\
              Free, fast inference\\
              14,400 req/day\\[0.2cm]
        \item \textbf{Together.AI} \muted{(backup)}\\
              \$5 free credit\\[0.2cm]
        \item \textbf{Ollama} \muted{(local)}\\
              Unlimited, runs in Colab
      \end{itemize}
    \end{column}

    \begin{column}{0.45\textwidth}
      \highlight{Infrastructure}\\[0.3cm]
      \begin{itemize}
        \item \textbf{Google Colab}\\
              Free T4 GPU\\
              All notebooks pre-configured\\[0.2cm]
        \item \textbf{HuggingFace}\\
              Models + datasets\\[0.2cm]
        \item \textbf{GitHub}\\
              All materials at\\
              \footnotesize\texttt{github.com/RJuro/}\\
              \texttt{unistra-nlp2026}
      \end{itemize}
    \end{column}
  \end{columns}
\end{frame}

% ═══════════════════════════════════════
% SLIDE 9: Project Tracks
% ═══════════════════════════════════════
\begin{frame}{Project Tracks}
  \framesubtitle{Pick one, work across Sprint 1 + Sprint 2}

  \begin{columns}[T]
    \begin{column}{0.48\textwidth}
      \highlight{A. Classification}\\
      \muted{Stance, sentiment, toxicity}\\
      \muted{Datasets: Moltbook, environmental claims}\\[0.4cm]

      \highlight{B. Topic Discovery}\\
      \muted{BERTopic + LLM annotation}\\
      \muted{Datasets: Moltbook, podcasts, Bluesky}
    \end{column}

    \begin{column}{0.48\textwidth}
      \highlight{C. Semantic Search}\\
      \muted{FAISS + cross-encoder reranking}\\
      \muted{Datasets: policy docs, academic papers}\\[0.4cm]

      \highlight{D. Structured Extraction}\\
      \muted{LLM \textrightarrow{} structured JSON \textrightarrow{} analysis}\\
      \muted{Datasets: SEC filings, podcast transcripts}
    \end{column}
  \end{columns}

  \vspace{0.5cm}
  \begin{center}
    \gold{Bring your own data} \muted{--- thesis-related encouraged!}
  \end{center}
\end{frame}

% ═══════════════════════════════════════
% SLIDE 10: Datasets
% ═══════════════════════════════════════
\begin{frame}{Datasets We'll Use}

  \footnotesize
  \begin{tabular}{@{}llll@{}}
    \toprule
    \textbf{Dataset} & \textbf{Size} & \textbf{Used In} & \textbf{Fun Factor} \\
    \midrule
    \highlight{dk\_posts} & 457 posts & NB01--03 & Reddit advice posts \\
    \highlight{Moltbook} & 44K posts & NB04 & AI agents on social media \\
    \highlight{Env. Claims} & binary & NB05 & Greenwashing detection \\
    \highlight{SciFact} & 300 docs & NB06--07 & Scientific retrieval \\
    \highlight{SEC Filings} & 1000s & Project D & Real financial data \\
    \highlight{Bluesky} & 2M posts & Project B & Platform migration \\
    \bottomrule
  \end{tabular}

  \vspace{0.4cm}
  \muted{All datasets available on HuggingFace or included in the repo.}
\end{frame}

% ═══════════════════════════════════════
% SLIDE 11: What You'll Build
% ═══════════════════════════════════════
\begin{frame}{What You'll Build}
  \framesubtitle{By Thursday afternoon}

  \begin{enumerate}
    \item A \highlight{baseline classifier} that actually works (TF-IDF)
    \item A \highlight{zero-shot LLM classifier} with structured output
    \item A \highlight{semantic search engine} with reranking
    \item A \highlight{fine-tuned language model} (Qwen3-4B)
    \item A \highlight{project pipeline} with honest evaluation
  \end{enumerate}

  \vspace{0.5cm}
  \begin{alertblock}{Sprint Deliverables}
    \textbf{Sprint 1} (Tue PM): Dataset + baseline + metric + 5 errors\\
    \textbf{Sprint 2} (Thu PM): Best pipeline + evaluation + model card
  \end{alertblock}
\end{frame}

% ═══════════════════════════════════════
% SLIDE 12: Ground Rules
% ═══════════════════════════════════════
\begin{frame}{Ground Rules}

  \begin{itemize}
    \item \highlight{Ask questions} --- there are no stupid ones
    \item \highlight{Help each other} --- peer learning is powerful
    \item \highlight{Break things} --- that's how you learn
    \item \highlight{Show your errors} --- error analysis > accuracy chasing
    \item \highlight{Use AI tools} --- Copilot, ChatGPT, Claude are all fair game
  \end{itemize}

  \vspace{0.5cm}
  \muted{Breaks every 90 minutes. Coffee is essential.}
\end{frame}

% ═══════════════════════════════════════
% SLIDE 13: Let's Start!
% ═══════════════════════════════════════
\begin{frame}[plain]
  \vfill
  \begin{center}
    {\Huge\color{cream} Let's start.}\\[0.8cm]
    \textcolor{coral}{\rule{4cm}{2pt}}\\[0.8cm]
    {\large\color{textmuted} Open NB01 in Colab}\\[0.3cm]
    {\footnotesize\color{textmuted}\texttt{colab.research.google.com/github/RJuro/\\unistra-nlp2026/blob/main/notebooks/NB01\_tfidf\_baselines.ipynb}}
  \end{center}
  \vfill
\end{frame}

\end{document}
